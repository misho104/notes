% Time-Stamp: <2015-06-13 22:33:42 misho>
\documentclass[10pt,a4paper]{article}
\pdfoutput=1
% -------------------------------------------------------- Page Styles
\voffset-1in
\oddsidemargin  .14\paperwidth
\evensidemargin .14\paperwidth
\marginparwidth .11\paperwidth
\textwidth      .72\paperwidth
\hoffset-1in
\topmargin  .05\paperheight
\headheight .02\paperheight
\headsep    .03\paperheight
\footskip   .07\paperheight
\textheight .76\paperheight
% ----------------------------------------------------------- Packages
\usepackage{amsmath,amssymb,url,cite,ifpdf,slashed,multirow,wrapfig,booktabs,color}
\usepackage[colorlinks=true,urlcolor=blue,citecolor=magenta]{hyperref}
%%%UNUSED%%% \usepackage{feynmp,enumerate}
\renewcommand\citepunct{,\penalty1000\hskip.13emplus.1emminus.1em\relax} % no line-break in \cite

\ifpdf                                     % For pdfLaTeX
  \usepackage{graphicx}                    %   usepackage without driver option
% \DeclareGraphicsRule{*}{mps}{*}{}        %   [For feynMP] Accept metapost graphics
\else                                      % For (p)LaTeX + dvipdfmx
  \usepackage[dvipdfmx]{graphicx}          %   usepackage with driver option
% \usepackage{mediabb}                     %   [For PDF graphics] Automatic PDF-bounding box
\fi

%%% Use AMS commands instead of default ones
\let\check\Check
\let\acute\Acute
\let\grave\Grave
\let\dot\Dot
\let\ddot\Ddot
\let\breve\Breve
\let\vec\Vec

% ---------------------------------------------------- For Sho's Notes
\usepackage{fancyhdr,color,soul}
\usepackage[hhmmss]{datetime}
\newdateformat{mydate}{\THEDAY\;\shortmonthname.\;\THEYEAR}
\pagestyle{fancy}
\renewcommand{\headrulewidth}{0pt}
\lhead{}
\rhead{\tt[\jobname~@~\mydate\today~\currenttime]}

% ----------------------------------------------------------- Commands
\newcommand{\un}[1]{\,{\rm #1}}
\newcommand{\TeV}{\,{\rm TeV}}
\newcommand{\GeV}{\,{\rm GeV}}
\newcommand{\MeV}{\,{\rm MeV}}
\newcommand{\keV}{\,{\rm keV}}
\newcommand{\fb}{\,{\rm fb}}
\newcommand{\pb}{\,{\rm pb}}
\newcommand{\invfb}{\,{\rm fb^{-1}}}
\newcommand{\invpb}{\,{\rm pb^{-1}}}

%%% SPECIFIC TO THIS ARTICLE
\makeatletter
\def\EE{\@ifnextchar-{\@@EE}{\@EE}}
\def\@EE#1{\ifnum#1=1 \times\!10 \else \times\!10^{#1}\fi}
\def\@@EE#1#2{\times\!10^{-#2}}
\makeatother
\renewcommand\thefootnote{*\arabic{footnote}}
\newcommand{\software}[1]{\texttt{#1}}
\newcommand{\mET}{{\slashed E}\s T}
%\newcommand{\vev}[1]{\left\langle #1\right\rangle}
\newcommand{\s}[1]{_{\rm #1}}
%\newcommand{\PL}{{\rm P_L}}
%\newcommand{\PR}{{\rm P_R}}
%\newcommand{\ii}{{\rm i}}
\newcommand{\dd}{{\rm d}}
%\newcommand{\ee}{{\rm e}}
\newcommand{\vc}[1]{{\boldsymbol{#1}}}
%\newcommand{\averaged}[1]{\left\langle{#1}\right\rangle}
\newcommand{\PT}{p\s{T}}
\newcommand{\vabs}[1]{\|#1\|}


%%% FOOTNOTE FOR TITLEPAGE
%\makeatletter
%\newcommand{\@authornote}[2]{{\def\thefootnote{\fnsymbol{footnote}}\setcounter{footnote}{#1}#2\setcounter{footnote}{0}}}
%\newcommand{\authornotemark}[1]{\@authornote#1{\addtocounter{footnote}{-1}\footnotemark}}
%\newcommand{\authornotetext}[2]{\@authornote#1{\footnotetext{#2}}}
%\makeatother

%%% EPRINT NUMBER TRICK
\makeatletter
\let\ORG@citex\@citex
\def\@citex[#1]#2{\@ifundefined{name@#2}{}{{\csname name@#2\endcsname}}\!\ORG@citex[#1]{#2}}
\makeatother

%%%%%%%%%%%%%%%%%%%%%%%%%%%%%%%%%%%%%%%%%%%%%%%%%%
% remove these commands when finalizing
\bibliographystyle{bst/referencelist}
\def\TODO#1{ {\bf ($\clubsuit$ #1 $\clubsuit$)} }
%%%%%%%%%%%%%%%%%%%%%%%%%%%%%%%%%%%%%%%%%%%%%%%%%%

\setcounter{secnumdepth}{2}
\begin{document}
\section{Spin determination at colliders}

\subsubsection{Lepton colliders}
Spin measurements in electron colliders are studied in \cite{Feng:1995zd}, \cite{Battaglia:2005zf}, \cite{Bhattacharyya:2005vm}, \cite{Bhattacherjee:2005qe}, \cite{Riemann:2005es}, and more.

\subsubsection[LHC, q-l-l chain]{LHC, $q$--$l$--$l$ chain}

\begin{itemize}
 \item \cite{Barr:2004ze} considers SUSY $\tilde q \to \tilde\chi^0_2 \to \tilde l \to \tilde\chi^0_1$ cascade decay to find jet--lepton angular correlation. NLO correction is found in \cite{Horsky:2008yi}. As applications,
\begin{itemize}
 \item \cite{Smillie:2005ar} and \cite{Datta:2005zs} to discriminate SUSY and UED by the method (the power is not very strong); the former mentions the production cross section can be used as a criterion; the latter utilizes the second KK mode production.

 \item \cite{Goto:2004cpa} to study L--R mixing of sleptons.
 \item \cite{Choi:2008pi} to study whether the gaugino is Majorana or Dirac.
 \item \cite{Hisano:2008ng} to determine electroweakino hierarchy.
 \item \cite{Gedalia:2009ym}, with third-generation initial squark, to discriminate SUSY/UED.
\end{itemize}

 \item \cite{Alves:2006df} provides a technique to assert $\tilde g$ is Majorana fermion, based on the chain $\tilde g\to\tilde b(\tilde b^*)\to\tilde \chi^0_2\to\tilde l\to\tilde\chi^0_1$, where we can determine whether $\tilde b$ or $\tilde b^*$ is produced by measuring the daughter $b$.
%
 \item \cite{Wang:2006hk} is for chargino/neutralino spin determination in  $\tilde q\to qW\tilde\chi$ decay channel. It especially provides for $X\to Y\to Z$-chain with $p_1p_2Z$ final states
 \begin{equation}
  \frac{\dd\Gamma}{\dd t_{12}}=a_0+\cdots+a_{2s}(t_{12})^{2s}
   \qquad\text{where}\quad t_{12} = (p_1+p_2)^2.
 \end{equation}
       This channel was studied later in \cite{Smillie:2006cd}.

 \item \cite{Kilic:2007zk} extended the work to scenarios in which primary and intermediate particles with any spins.

 \item A general framework for the spin determination in $D\to Cq\to Bql\to Aqll$ chain is found in \cite{Athanasiou:2006ef} (and \cite{Athanasiou:2006hv}) with plenty of analytic formulae.
       More practical analyses are found in \cite{Burns:2008cp}.

\end{itemize}

\subsubsection{LHC, other chains}
\begin{itemize}
 \item\cite{Barr:2005dz} focuses on $\tilde l \tilde l^*$ DY production (short-lived) to discriminate SUSY/UED.
      They propose a variable $\tanh(\eta_{l^+}-\eta_{l^-})/2$, which is boost-invariant and thus can be useful.

 \item\cite{Alwall:2007ed} considers $\tilde W^+\tilde W^+jj$ VBF production, mediated by Majorana neutralino, and detemines the spin of $\tilde W^+$ (long-lived) by $m_{jj}$ of the forward jets.

 \item\cite{Alves:2007xt} is the same analysis for $\tilde b\tilde b^*$ DY production.

 \item\cite{Csaki:2007xm} uses $m_{jj}$ distribution of $\tilde g\to jj\tilde\chi^0_1$ process to discriminate SUSY/UED.

 \item\cite{Ehrenfeld:2009rt} uses chains with $\gamma$, i.e., $\tilde\chi^0_2\to\tilde l\to\tilde\chi^0_1\to\tilde G$ ($ll\gamma$) and $\tilde q\to \tilde\chi^0_1\to\tilde G$ ($q\gamma$) to discriminate spins, but the power is not significant.
      
 \item \cite{Kim:2011zzg} discusses mass and spin measurement in $\tilde q\s L\to qZ(\to ll)\tilde\chi^0_1$ chain, where intermediate $\tilde\chi^0_2$ can be polarized which results in the angular distribution of $Z$.
 \end{itemize}


\subsubsection{LHC, indirect spin measurement}
\begin{itemize}
 \item \cite{Datta:2005vx} focuses on same-sign di-lepton events and utilizes its cross section for SUSY/UED discrimination.
 \item\cite{Meade:2006dw} discusses $t'$ pair-production, where $t'$ is a particle decaying into $t+\text{missing}$, and its mass and spin determination. The mass is determined by cross section in a two-fold way (lower for scalar, higher for fermion), and then, as an indirect spin determination, the angular distribution {\em in the lab frame} is utilized to distinguish the two-fold.
      A detector simulation is considered in \cite{Hallenbeck:2008hf}.
 \item \cite{Kane:2008kw} discusses crosssection-based spin measurement in colored pair production.
\end{itemize}

% \item \cite{Rajaraman:2007ae} initiates spin measurement of LLCP.


\subsubsection{Others}
\begin{itemize}
 \item \cite{Alves:2008up} determines the spin of a resonance $V^+$ in $pp\to V^+jj$ (VBF) $\to WZjj$ process, possible in Higgsless models, where $\cos\theta^*_{ll}$ is utilized.
 \item \cite{Graesser:2008qi} collects possible correlations in $b$--$\tau$, $b$--$l$, and $\tau$--$l$, including possible mixings, but merely theoretical calculation.
\end{itemize}





%\cite{Rajaraman:2007ae}
%\cite{Feng:2015wqa}
\bibliography{bib/lhc_spin_measurement}
\end{document}

%%% Local Variables:
%%% TeX-master: t
%%% End:
