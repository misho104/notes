% Time-Stamp: <2016-11-15 14:06:18 misho>
\documentclass[10pt,a4paper]{article}
%$\pdfoutput=1
% -------------------------------------------------------- Page Styles
\voffset-1in
\oddsidemargin  .14\paperwidth
\evensidemargin .14\paperwidth
\marginparwidth .11\paperwidth
\textwidth      .72\paperwidth
\hoffset-1in
\topmargin  .05\paperheight
\headheight .02\paperheight
\headsep    .03\paperheight
\footskip   .07\paperheight 
\textheight .76\paperheight
% ----------------------------------------------------------- Packages
\usepackage{amsmath,amssymb,url,cite,booktabs,cancel,hyperref,soul}
%%%UNUSED%%% \usepackage{ifpdf,slashed,multirow,wrapfig,cancel,feynmp,enumerate,hyperref}
\renewcommand\citepunct{,\penalty1000\hskip.13emplus.1emminus.1em\relax} % no line-break in \cite

%\ifpdf                                     % For pdfLaTeX
  \usepackage{graphicx}                    %   usepackage without driver option
% \DeclareGraphicsRule{*}{mps}{*}{}        %   [For feynMP] Accept metapost graphics
%\else                                      % For (p)LaTeX + dvipdfmx
% \usepackage[dvipdfmx]{graphicx}          %   usepackage with driver option
% \usepackage{mediabb}                     %   [For PDF graphics] Automatic PDF-bounding box
%\fi

%%% Use AMS commands instead of default ones
\let\check\Check
\let\acute\Acute
\let\grave\Grave
\let\dot\Dot
\let\ddot\Ddot
\let\breve\Breve
\let\vec\Vec

\renewcommand{\thefootnote}{$\ast$\arabic{footnote}}

% ----------------------------------------------------------- Commands
\newcommand{\un}[1]{\,{\rm #1}}
\newcommand{\TeV}{\,{\rm TeV}}
\newcommand{\GeV}{\,{\rm GeV}}
\newcommand{\MeV}{\,{\rm MeV}}
\newcommand{\keV}{\,{\rm keV}}
\newcommand{\ab}{\,{\rm ab}}
\newcommand{\fb}{\,{\rm fb}}
\newcommand{\pb}{\,{\rm pb}}
\newcommand{\ifb}{\,{\rm fb^{-1}}}
\newcommand{\ipb}{\,{\rm pb^{-1}}}
\newcommand{\iab}{\,{\rm ab^{-1}}}
\newcommand{\Order}{\mathcal O}
\newcommand{\pmat}[1]{\begin{pmatrix}#1\end{pmatrix}}
%%% SPECIFIC TO THIS ARTICLE
\newcommand{\w}[1]{_{\rm #1}}
\newcommand{\ii}{{\rm i}}
\newcommand{\dd}{{\rm d}}
\newcommand{\ee}{{\rm e}}
\newcommand{\vc}[1]{{\boldsymbol{#1}}}
\newcommand{\Br}{\mathop{\mathrm {Br}}}
%%%%%%%%%%%%%%%%%%%%%%%%%%%%%%%%%%%%%%%%%%%%%%%%%%
% remove these commands when finalizing
\bibliographystyle{referencelist}
\usepackage{fancyhdr,color,soul}
\usepackage[hhmmss]{datetime}
\newdateformat{mydate}{\THEDAY\;\shortmonthname.\;\THEYEAR}
\pagestyle{fancy}
\renewcommand{\headrulewidth}{0pt}
\lhead{}
\rhead{\tt[\jobname~@~\mydate\today~\currenttime]}
\newcommand\TODO[1]{ {\bf ($\clubsuit$ #1 $\clubsuit$)} }
\makeatletter
\protected\def\EE{\@ifnextchar-{\@@EE}{\@EE}}
\protected\def\@EE#1{\ifnum#1=1 \!\times\!10 \else \!\times\!10^{#1}\fi}
\protected\def\@@EE#1#2{\!\times\!10^{-#2}}
\let\ORG@citex\@citex
\def\@citex[#1]#2{\@ifundefined{name@#2}{\!}{{\csname name@#2\endcsname}}\!\ORG@citex[#1]{#2}}
\makeatother
%%%%%%%%%%%%%%%%%%%%%%%%%%%%%%%%%%%%%%%%%%%%%%%%%%
\begin{document}
\subsection{Three-body phase space}
In the notation of \url{http://hitoshi.berkeley.edu/233B/phasespace.pdf},
\begin{equation}
   \int\dd\Phi_3
   =\int
   \frac{\dd s_{23}}{2\pi}
   \frac{\dd \cos\theta_1}{2}
   \frac{\dd \phi_1}{2\pi}
   \frac{\bar\beta_1(\frac{m_1^2}{s},\frac{s_{23}}{s})}{8\pi}
   \frac{\dd\cos\hat\theta_{23}}{2}
   \frac{\dd\hat\phi_{23}}{2\pi}
   \frac{\bar\beta_{23}(\frac{m_2^2}{s_{23}},\frac{m_3^2}{s_{23}})}{8\pi},
\end{equation}
where $(\theta_1,\phi_1)$ is the solid angle for $p_1$ (thus $p_{23}$), $(\hat\theta_{23},\hat\phi_{23})$ is that for $p_2$ (thus $p_3$) evaluated in the rest frame of $p_{23}$, $s_{ij}$ are equal to $m_{ij}^2$ in the traditional Dalitz language, and
\begin{align}
 \bar\beta(x,y)&:=\sqrt{1-2(x+y)+(x-y)^2};
 &
 \bar\beta_1 &:= \bar\beta\left(\frac{m_1^2}{s},\frac{s_{23}}{s}\right),
 &
 \bar\beta_{23} &:= \bar\beta\left(\frac{m_2^2}{s_{23}},\frac{m_3^2}{s_{23}}\right)
 \end{align}
(the subscript of $\bar\beta$ is (perhaps) just redundancy for clarity).

For a spherically-symmetric integrand $|\mathcal M|^2$,
\begin{itemize}
 \item we can drop $\dd\cos\theta_1/2$ and $\dd\phi_1/2\pi$,
 \item we can take $\hat\theta_{23}$ relative to $-p_1=p_{23}$ (in $p_1+p_2+p_3$ rest frame) to drop $\dd\hat\phi_{23}/2\pi$,
\end{itemize}
to get
\begin{equation}
   \int\dd\Phi_3
   =\int
   \frac{\dd s_{23}}{2\pi}
   \frac{\bar\beta_1(\frac{m_1^2}{s},\frac{s_{23}}{s})}{8\pi}
   \frac{\dd\cos\hat\theta_{23}}{2}
   \frac{\bar\beta_{23}(\frac{m_2^2}{s_{23}},\frac{m_3^2}{s_{23}})}{8\pi}.
\label{eq:dphi3reduced}
\end{equation}

\subsubsection{Energy fractions}
Eq.~\eqref{eq:dphi3reduced} is rewritten by the energy fractions
\begin{equation}
 x_i = \frac{E_i}{\sqrt{s}/2}
\end{equation}
with using the relations
\begin{align}
 s &= (E_1+E_2+E_3)^2 = (\hat E_1+\hat E_2+\hat E_3)^2-|\hat {\vc p}_1|^2,
 &
 s_{23} &= (\sqrt{s}-E_1)^2-|\vc p_1|^2 = (\hat E_2+\hat E_3)^2.
\end{align}
From the second equation, we know $s_{23}=s+m_1^2-s x_1$ and
\begin{equation}
   \int\dd\Phi_3
   =\int
   \frac{s \dd x_1}{2\pi}
   \frac{\bar\beta_1(\frac{m_1^2}{s},\frac{s_{23}}{s})}{8\pi}
   \frac{\dd\cos\hat\theta_{23}}{2}
   \frac{\bar\beta_{23}(\frac{m_2^2}{s_{23}},\frac{m_3^2}{s_{23}})}{8\pi}.
\end{equation}
where
\begin{equation}
 x_1=\left[\frac{2m_1}{\sqrt s}, 1+\frac{m_1^2-(m_2+m_3)^2}{s}\right].
\end{equation}

The rest frame of $p_{23}$ is now fixed by $x_1$. Setting the $z$-axis as the direction of $\vc p_1$,
\begin{align}
 p_1 &= \frac{\sqrt s}{2}\pmat{x_1\\ 0 \\ 0 \\s},
& q_{23} = \frac{\sqrt s}{2}\pmat{x_1\\ 0 \\ 0 \\s}
\end{align}

\subsubsection{Dalitz plot}
In PDG Review, Eq.~\eqref{eq:dphi3reduced} is expressed by
\begin{equation}
   \int\dd\Phi_3
   =\int
   \frac{\dd s_{12}}{2\pi}
   \frac{\bar\beta_3(\frac{m_3^2}{s},\frac{s_{12}}{s})}{8\pi}
   \frac{\dd\cos\hat\theta_{12}}{2}
   \frac{\bar\beta_{12}(\frac{m_1^2}{s_{12}},\frac{m_2^2}{s_{12}})}{8\pi},
\end{equation}
where variables with hats are in the rest frame of $p_{12}$. Then $\hat\theta_{12}$ is converted to $s_{23}$ by
\begin{equation}
\begin{split}
 s_{23}
 &= (p_2+p_3)^2
\\&=
 m_2^2 + m_3^2 +2\left(\hat E_2\hat E_3 - \sqrt{\left(\hat E_2^2-m_2^2\right)\left(\hat E_3^2-m_3^2\right)}\cos\hat\theta_{12}\right),
\end{split}\end{equation}
where note that $\hat\theta_{12}$ is defined by the angle between $-\vc p_3$ and $\vc p_1$, and therefore the angle between $\vc p_2$ and $\vc p_3$. $\hat E_{i}$ are known by
\begin{align}
 (\hat E_1+\hat E_2+\hat E_3)^2 - |\hat {\vc p}_3|^2 &= s,
 &
 (\hat E_1+\hat E_2)^2 &= s_{12},
\end{align}
as
\begin{align}
  E_1&=\frac{s_{12}+m_1^2-m_2^2}{2\sqrt{s_{12}}},
 &E_2&=\frac{s_{12}-m_1^2+m_2^2}{2\sqrt{s_{12}}},
 &E_3&=\frac{s-s_{12}-m_3^2}{2\sqrt{s_{12}}}.
\end{align}
Thus,
\begin{equation}
    \int\dd\Phi_3
   =\int
   \frac{\dd s_{12}}{2\pi}
   \frac{\bar\beta_3(\frac{m_3^2}{s},\frac{s_{12}}{s})}{8\pi}
   \frac{\dd s_{23}}{4\sqrt{(\hat E_2^2-m_2^2)(\hat E_3^2-m_3^2)}}
   \frac{\bar\beta_{12}(\frac{m_1^2}{s_{12}},\frac{m_2^2}{s_{12}})}{8\pi}
=   \frac{1}{128\pi^3s}\int\dd s_{12}\dd s_{23}.
\end{equation}

\end{document}